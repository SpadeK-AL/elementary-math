
\section{圆}
\label{sec:circle}

\subsection{圆幂}
\label{sec:power-of-circle}

早在中学数学中,我们就已经知道切割线定理和相交弦定理,而今我们将以统一的观点来看待这两个定理,这两个定理的共同点是,从一个定点作两条直线与定圆相交(相切可以视为极端情形),在其中每一条直线上,从定点出发到两个交点的两条线段的长度之积,都是相同的,也就是说这个乘积与具体的直线无关,仅与这个点与圆有关,于是我们便用如下的圆幂定理来取代切割线定理和相交弦定理:

\begin{theorem}[圆幂定理]
  过平面上一个定点引直线与一个定圆相交得到两个交点(相切时两个交点重合),则从该定点出发分别到达两个交点的两条线段长度之积是与直线位置无关的定值。
\end{theorem}

那么接下来就要问,这个定值是多少?对于切割线定理而来,显然这个定点就是该点到圆的切线长的平方,而相交弦定理来说,让通过此点的弦以该点为中点,则显见这个定值是半径的平方与弦心距的平方之差,于是我们引入如下定义:

\begin{definition}
  对于平面上的一个定点和一个定圆,称定点到圆心距离的平方减去半径的平方所得的差为该点对该圆的 \emph{幂}。
\end{definition}

由定义,圆外的点对圆的幂为正,圆内的点对圆的幂为负,圆上的点对此圆的幂为零,于是圆幂定理中的定值,便是定点对定圆的幂的绝对值。

\subsection{反演变换}
\label{sec:transformation-of-inversion}



\subsection{极点与极线}
\label{sec:pole-and-polar}


\subsection{根轴与根心}
\label{sec:radical-axis-and-center}





%%% Local Variables:
%%% mode: latex
%%% TeX-master: "../../elementary-math-note"
%%% End:
