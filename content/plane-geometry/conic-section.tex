
\section{圆锥曲线}
\label{sec:conic-section}

\begin{theorem}
  椭圆$\dfrac{x^2}{a^2}+\dfrac{y^2}{b^2}=1(a>0,b>0)$上任一点$P(x_0,y_0)$处的切线方程是$\dfrac{x_0x}{a^2}+\dfrac{y_0y}{b^2}=1$.
\end{theorem}

\begin{proof}[证明]
在这直线上任取一点 $T(x_T,y_T)$,有:
\begin{equation}
(\frac{x_0^2}{a^2}+\frac{y_0^2}{b^2})+(\frac{x_T^2}{a^2}+\frac{y_T^2}{b^2}) \geqslant 2(\frac{x_0x_T}{a^2}+\frac{y_0y_T}{b^2})=2
\end{equation}
所以得到:
\begin{equation}
\frac{x_T^2}{a^2}+\frac{y_T^2}{b^2} \geqslant 1
\end{equation}
这表明直线\ref{eq:tangent}上除点$P$外任何一点都在椭圆外,与椭圆只有$P$一个交点,所以它理所当然就是点$P$处的切线方程。
\end{proof}

\begin{theorem}
  双曲线$\dfrac{x^2}{a^2}-\dfrac{y^2}{b^2}=1(a>0,b>0)$上任一点$P(x_0,y_0)$处的切线方程是$\dfrac{x_0x}{a^2}-\dfrac{y_0y}{b^2}=1$.
\end{theorem}

\begin{proof}[证明]
  在直线$\dfrac{x_0x}{a^2}-\dfrac{y_0y}{b^2}=1$上任取一点$Q(x_1,y_1)$,有
  \begin{equation}
    \label{eq:4hdks85hdks018}
   \frac{x_0^2}{a^2}-\frac{y_0^2}{b^2}=1, \  \frac{x_0x_1}{a^2}+\frac{y_0y_1}{b^2}=1 
  \end{equation}
  我们将证明
  \begin{equation}
    \label{eq:hss8wsk13hcfsud0wurg}
    \frac{x_1^2}{a^2}-\frac{y_1^2}{b^2} \leqslant 1  
  \end{equation}
  且等号仅在$y_1=y_0$时成立,这样一来,点$P(x_0,y_0)$就是直线与双曲线的唯一公共点,即为切线。

  记
  \[ r=\frac{x_0}{a}+\frac{y_0}{b}, \  s=\frac{x_0}{a}-\frac{y_0}{b} \]
  以及
  \[ u=\frac{x_1}{a}+\frac{y_1}{b}, \  v=\frac{x_1}{a}-\frac{y_1}{b} \]
  那么\autoref{eq:4hdks85hdks018}即为$rs=1$以及$\dfrac{rv+su}{2}=1$,而要证明的\autoref{eq:hss8wsk13hcfsud0wurg}即是$uv\leqslant 1$.

  由$4=(rv+su)^2=r^2v^2+s^2u^2+2rsuv \geqslant 4rsuv= 4uv$即证得$uv \leqslant 1$,定理得证。
\end{proof}

\begin{theorem}
  抛物线$y^2=2px(p>0)$上任一点$P(x_0,y_0)$处的切线方程是$y_0y=p(x+x_0)$.
\end{theorem}

\begin{proof}[证明]
  在直线$y_0y=p(x+x_0)$上任取一点$Q(x_1,y_1)$,则
  \[ y_1^2+y_0^2 \geqslant 2y_1y_0=2p(x_1+x_0) \]
  而$y_0^2=2px_0$,所以得$y_1^2 \geqslant 2px_1$,且等号仅在$y_1=y_0$时成立,这表明点$P$是该直线与抛物线的唯一公共点,即为切线。
\end{proof}

%%% Local Variables:
%%% mode: latex
%%% TeX-master: "../../elementary-math-note"
%%% End:
