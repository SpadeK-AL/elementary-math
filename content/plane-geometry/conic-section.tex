
\section{圆锥曲线}
\label{sec:conic-section}

\begin{theorem}
  椭圆$\dfrac{x^2}{a^2}+\dfrac{y^2}{b^2}=1(a>0,b>0)$上任一点$P(x_0,y_0)$处的切线方程是$\dfrac{x_0x}{a^2}+\dfrac{y_0y}{b^2}=1$.
\end{theorem}

\begin{proof}[证明]
在这直线上任取一点 $T(x_T,y_T)$,有:
\begin{equation}
\left(\frac{x_0^2}{a^2}+\frac{y_0^2}{b^2}\right)+\left(\frac{x_T^2}{a^2}+\frac{y_T^2}{b^2}\right) \geqslant 2\left(\frac{x_0x_T}{a^2}+\frac{y_0y_T}{b^2}\right)=2
\end{equation}
所以得到:
\begin{equation}
\frac{x_T^2}{a^2}+\frac{y_T^2}{b^2} \geqslant 1
\end{equation}
这表明直线\ref{eq:tangent}上除点$P$外任何一点都在椭圆外,与椭圆只有$P$一个交点,所以它理所当然就是点$P$处的切线方程。
\end{proof}

\begin{proof}[证明二]
  设$P(x_0,y_0)$处对应椭圆离心角$\theta_0$,将椭圆参数方程
\[ 
\begin{cases}
x & = a \cos{\theta} \\
y & = b \sin{\theta}
\end{cases}
\]
代入直线方程
\[ \frac{x_0x}{a^2} + \frac{y_0y}{b^2} = 1 \]
同时用$x_0=a\cos{\theta_0},y_0=b\sin{\theta_0}$替换掉$x_0$和$y_0$得
\[ \cos{\theta_0}\cos{\theta}+\sin{\theta_0}\sin{\theta}=1 \]
即
\[ \cos{(\theta-\theta_0)} = 1 \]
显然,只有$\theta$与$\theta_0$相差$\pi$的偶数倍,才能成为椭圆与直线的交点,而这些交点实际上都与$P(x_0,y_0)$是同一点,于是即证。
\end{proof}

\begin{theorem}
  双曲线$\dfrac{x^2}{a^2}-\dfrac{y^2}{b^2}=1(a>0,b>0)$上任一点$P(x_0,y_0)$处的切线方程是$\dfrac{x_0x}{a^2}-\dfrac{y_0y}{b^2}=1$.
\end{theorem}

\begin{proof}[证明]
  在直线$\dfrac{x_0x}{a^2}-\dfrac{y_0y}{b^2}=1$上任取一点$Q(x_T,y_T)$,有
  \begin{equation}
    \label{eq:4hdks85hdks018}
   \frac{x_0^2}{a^2}-\frac{y_0^2}{b^2}=1, \  \frac{x_0x_T}{a^2}-\frac{y_0y_T}{b^2}=1 
  \end{equation}
  我们将证明
  \begin{equation}
    \label{eq:hss8wsk13hcfsud0wurg}
    \frac{x_T^2}{a^2}-\frac{y_T^2}{b^2} \leqslant 1  
  \end{equation}
  且等号仅在$y_T=y_0$时成立,这样一来,点$P(x_0,y_0)$就是直线与双曲线的唯一公共点,即为切线。

  记
  \[ r=\frac{x_0}{a}+\frac{y_0}{b}, \  s=\frac{x_0}{a}-\frac{y_0}{b} \]
  以及
  \[ u=\frac{x_T}{a}+\frac{y_T}{b}, \  v=\frac{x_T}{a}-\frac{y_T}{b} \]
  那么\autoref{eq:4hdks85hdks018}即为$rs=1$以及$\dfrac{rv+su}{2}=1$,而要证明的\autoref{eq:hss8wsk13hcfsud0wurg}即是$uv\leqslant 1$.

  由$4=(rv+su)^2=r^2v^2+s^2u^2+2rsuv \geqslant 4rsuv= 4uv$即证得$uv \leqslant 1$,定理得证。
\end{proof}

网友kuing提出,椭圆和双曲线有统一的类似的漂亮写法,因为
\begin{align*}
  (a^2+b^2)(u^2+v^2) & =  (au+bv)^2+(av-bu)^2 \geqslant (au+bv)^2 \\
  (a^2-b^2)(u^2-v^2) & =  (au-bv)^2-(av-bu)^2 \leqslant (au-bv)^2 
\end{align*}
所以
\begin{align*}
  \left( \frac{x_0^2}{a^2}+\frac{y_0^2}{b^2} \right) \left( \frac{x_T^2}{a^2}+\frac{y_T^2}{b^2} \right) & \geqslant  \left( \frac{x_0x_T}{a^2}+\frac{y_0y_T}{b^2} \right)^2 \\
  \left( \frac{x_0^2}{a^2}-\frac{y_0^2}{b^2} \right) \left( \frac{x_T^2}{a^2}-\frac{y_T^2}{b^2} \right) & \leqslant  \left( \frac{x_0x_T}{a^2}-\frac{y_0y_T}{b^2} \right)^2 
\end{align*}
如此证法将更为漂亮。

\begin{theorem}
  抛物线$y^2=2px(p>0)$上任一点$P(x_0,y_0)$处的切线方程是$y_0y=p(x+x_0)$.
\end{theorem}

\begin{proof}[证明]
  在直线$y_0y=p(x+x_0)$上任取一点$Q(x_T,y_T)$,则
  \[ y_T^2+y_0^2 \geqslant 2y_Ty_0=2p(x_T+x_0) \]
  而$y_0^2=2px_0$,所以得$y_T^2 \geqslant 2px_T$,且等号仅在$y_T=y_0$时成立,这表明点$P$是该直线与抛物线的唯一公共点,即为切线。
\end{proof}

%%% Local Variables:
%%% mode: latex
%%% TeX-master: "../../elementary-math-note"
%%% End:
