
\section{同余}
\label{sec:congruences}

\begin{definition}
  给定一个正整数$m$,称为\emph{模},$a$和$b$是两个正整数,如果用$m$去除$a$、$b$所得余数相同,则称$a$、$b$模$m$\emph{同余},记作$a \equiv b \pmod{m}$,如果余数不同,则称$a$、$b$模$m$不同余,记作$a \not \equiv b \pmod{m}$.
\end{definition}

容易验证,同余满足以下三条性质:
\begin{enumerate}
\item $a \equiv a \pmod{m}$.
\item 若 $a \equiv b \pmod{m}$,则也有 $b \equiv a \pmod{m}$.
\item 若$a \equiv b \pmod{m}$且$b \equiv c \pmod{m}$,则$a \equiv c \pmod{m}$.
\end{enumerate}
这表明,同余是一种等价关系。

\begin{theorem}
  整数$a$、$b$对模$m$同余的充分必要条件是$m \mid (a-b)$.
\end{theorem}

\begin{proof}[证明]
  如果$a \equiv b \pmod{m}$,则存在两个整数$k_1$和$k_2$和整数$r(0 \leqslant r < m)$,使得$a=k_1m+r$,$b=k_2m+r$,于是$m \mid (a-b)$,反之,若$m \mid (a-b)$,设$a-b=mk$,则显然有$a \equiv b \pmod{m}$.
\end{proof}

这定理的意义在于,我们可以把同余问题转化为整除问题来解决。

\begin{theorem}
根据整除理论,同余还有以下性质:
\begin{enumerate}
\item 若$a_1 \equiv b_1 \pmod{m}$, $a_2 \equiv b_2 \pmod{m}$,则$(a_1\pm a_2) \equiv (b_1 \pm b_2) \pmod{m}$.
\item 若$(a+b) \equiv c \pmod{m}$,则$a \equiv (c-b) \pmod{m}$.
\item 若$a_1 \equiv b_1 \pmod{m}$, $a_2 \equiv b_2 \pmod{m}$,则$a_1a_2 \equiv b_1b_2 \pmod{m}$. 特殊情况是,若$a \equiv b \pmod{m}$,$k$是一个整数,则$ak \equiv bk \pmod{m}$.
\end{enumerate}
\end{theorem}

上述定理容易验证,此外还有更一般的结论:
\begin{theorem}
  若$A_{s_1,s_2,\ldots,s_n} \equiv B_{s_1,s_2,\ldots,s_n} \pmod{m}$,且$x_i \equiv y_i \pmod{m}$,$i=1,2,\ldots,n$,则
  \[ \sum_{s_1,s_2,\ldots,s_n} A_{s_1,s_2,\ldots,s_n} x_1^{s_1}x_2^{s_2} \cdots x_n^{s_n} \equiv =  \sum_{s_1,s_2,\ldots,s_n} B_{s_1,s_2,\ldots,s_n} y_1^{s_1}y_2^{s_2} \cdots y_n^{s_n} \pmod{m} \]
\end{theorem}

特殊情况是,若$a_i \equiv b_i \pmod{m}$,则
\[ a_nx^n+a_{n-1}x^{n-1}+\cdots a_1x+a_0 \equiv b_nx^n+b_{n-1}x^{n-1}+\cdots b_1x+b_0 \pmod{m} \]

以上性质和定理都不难证明。

\begin{theorem}
  同余还具有以下一些性质:
  \begin{enumerate}
  \item 若$a \equiv b \pmod{m}$,且$a=a_1d$,$b=b_1d$,$(d,m)=1$,则$a_1 \equiv b_1 \pmod{m}$.
  \item 若$a \equiv b \pmod{m}$,$k$是一个正整数,则$ak \equiv bk \pmod{mk}$.
  \item 若$a \equiv b \pmod{m}$,$d$是$a$、$b$及$m$的任一正的公因数,则$\frac{a}{d} \equiv \frac{b}{d} \pmod{\frac{m}{d}}$. 
  \item 若$a \equiv b (\mod m_i),i=1,2,\ldots,n$,则$a \equiv b \pmod{[m_1,m_2,\ldots,m_n]}$.
  \item 若$a \equiv b \pmod{m}$,$d$是$m$的一个正的因数,则$a \equiv b \pmod{d}$.
  \item 若$a \equiv b \pmod{m}$,则$(a,m)=(b,m)$,进而有,若$d$能整除$m$及$a$、$b$其中之一,则也一定能整除另一个。
  \end{enumerate}
\end{theorem}

\begin{proof}[证明]
  1. 由$a \equiv b \pmod{m}$得$m \mid (a-b)$,即$m \mid d(a_1-b_1)$,而$(m,d)=1$,所以$m \mid (a_1-b_1)$,即$a_1 \equiv b_1 \pmod{m}$.

  2.3.4.5. 这四条都是显然的.

  6. 由$a-b=km$知$a$与$m$的公因数集合,跟$b$与$m$的公因数集合相同,因此结论成立。
\end{proof}

同余可以用来简单的判别一个(较大的)整数是否可被另一整数整除。
\begin{theorem}
  十进制整数$a=a_n10^n+a_{n-1}10^{n-1}+\cdots+10a_1+a_0(0 \leqslant a_i \leqslant 9, a_n \neq 0)$能被$3$整除的充分必要条件是$3 \mid \sum_{i=0}^na_i$.
\end{theorem}

\begin{theorem}
  十进制整数$a=a_n10^n+a_{n-1}10^{n-1}+\cdots+10a_1+a_0(0 \leqslant a_i \leqslant 9, a_n \neq 0)$能被$7$(或$11$,或$13$)整除的充分必要条件是$7$(或$11$,或$13$)能够整除$\sum_{i=0}^n(-1)^ia_i$.
\end{theorem}

此外,还有所谓的弃九法,这是用来检查乘法运算的结果是否正确的一种简便方法,但它不能保证完全正确,既是说,通过检查的不一定是正确的,但是通不过检查的一定是错误的。假定有如下的十进制表示
\[ a = a_n10^n+a_{n-1}10^{n-1}+\cdots+a_0 \]
以及
\[ b = b_m10^m+b_{m-1}10^{m-1}+\cdots+b_0 \]
假定我们计算得出的乘积结果是(待检查的结果)
\[ P = c_l10^l+c_{l-1}10^{l-1}+\cdots+c_0 \]
那么如果
\[ \left( \sum_{i=0}^n a_i \right) \left( \sum_{i=0}^m b_i \right) \not \equiv \sum_{i=0}^l c_i (\mod 9) \]
那么我们求得的结果$P$一定是错误的,但是如果上式的同余式成立了也不一定是正确的,只是这个机会比较小,假定错误结果对9的余数是均匀随机,那么这种情况发生的概率是1/10,所以方法还是比较有效的.


\subsection{同余及其性质}
\label{sec:congruences-and-its-properties}

\subsection{剩余类与完全剩余系}
\label{sec:residue-and-complete-set-of-residues}

\subsection{欧拉函数、欧拉定理与费马小定理}
\label{sec:euler-function-and-euler-theorem-and-fermat-small-theorem}

\subsection{RSA加密与解密}
\label{sec:rsa-encrypt-decrypt}

\subsection{一次同余式与中国剩余定理}
\label{sec:congruence-of-first-degree-and-chinese-remainder-theorem}


%%% Local Variables:
%%% mode: latex
%%% TeX-master: "../../elementary-math-note"
%%% End:
