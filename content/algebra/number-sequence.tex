
\section{数列}
\label{sec:number-sequence-general}

\subsection{数列概念}

数列就是一个数的序列,可以是有限序列也可以是无限序列,按下标记法可以写成$a_1,a_2,\ldots$,有时为了方便也可以让下标从零开始$a_0,a_1,a_2,\ldots$,甚至作为无穷序列时,还可以把下标扩展到负整数:$\cdots,a_{-2},a_{-1},a_0,a_1,a_2,\cdots$,成为一个双向无穷数列,如果有一个公式可以把数列的每一项表成它的下标的函数,就称此公式为该数列的\emph{通项公式}。数列本质上是定义在整数集或者正整数集上的函数。

\begin{example}
  任一实数在十进制下可以表为
  \[ a_na_{n-1}\cdots a_1a_0.a_{-1}a_{-2}\cdots \]
  其中$a_i$取值范围是$0,1,\ldots,9$,这便可以视为一个数列,而且是单侧无穷数列,而这个实数的值是
  \[ a_n\cdot 10^n + a_{n-1} \cdot 10^{n-1} + \cdots + a_1 \cdot 10 + a_0 \cdot 10^0 + a_{-1} \cdot 10^{-1} + a_{-2} \cdot 10^{-2} +\cdots \]
\end{example}

除了通项公式之外,\emph{递推公式}也是研究数列的一个重要手段,所谓递推公式是指一个联系着数列中相邻若干项的一个公式,例如可以这样定义一个数列,约定$a_1=1$,$a_2=1$,以后的每一项都由它前面的两项之和来定义,即$a_{n+2}=a_{n+1}+a_n(n \geqslant 1)$,显然这公式唯一的确定了数列的每一项,也就确定了数列本身,它就是有名的 \emph{斐波那契(Fibonacci)数列},后文还会讨论它。

\subsection{等差数列与等比数列}

等差数列和等比数列是最常见的两种数列。

\begin{definition}
 如果一个数列$\ldots,a_{-2},a_{-1},a_0,a_1,a_2,\ldots$中,任一项减去它前一项所得差值都相同,即存在常数$d$,对任意整数$n$都有$a_n-a_{n-1}=d$,则称此数列为 \emph{等差数列},而常数$d$称为此等差数列的\emph{公差}.
\end{definition}

显然,正整数数列$\ldots,-2,-1,0,1,2,\ldots$是公差为1的等差数列,任何项都取相同值的\emph{常数数列}则是公差为零的等差数列。

等差数列的递推公式也可以用$a_{n+1}+a_{n-1}=2a_n$来刻画,而且其中不出现公差,仅仅通过项的关系来就定义出等差数列,这也表明,数列的递推公式可以不唯一。

\begin{theorem}
对于等差数列中的任意两项(无论顺序)有
\[ a_n-a_m=(n-m)d \]
\end{theorem}

\begin{proof}[证明]
证明是容易的,只要证明$n\geqslant m$的情况即可(否则可以反过来相减),首先可以验证$n$与$m$相等时该等式成立,而后有
\[ a_n-a_m=\sum_{k=m}^{n-1}(a_{k+1}-a_k)=(n-m)d \]
当然,对差值$n-m$施行归纳法也是可行的。
\end{proof}

假定我们知道$a_0$和公差$d$,在上式中取$m=0$,我们就得到等差数列的通项公式
\begin{theorem}
  等差数列的每一项可以由$a_0$及公差$d$表为
\[ a_n=a_0+nd \]
这便是等差数列的通项公式.
\end{theorem}

当然通项并不一定非得使用$a_0$来表示,$a_n=a_1+(n-1)d$,$a_n=a_{100}+(n-100)d$也都是通项。

易见这通项是关于$n$的一次函数,实际上容易证明,如果数列的通项可表成关于下标的一次函数,则它必然是等差数列。

现在考虑对等差数列进行某种求和,如果从$a_0$开始,向正下标或者负下标进行累加,$n$是任意一个整数(无论正负),记$S_n=\sum_{i=0}^{n}a_n$,则
\[ S_n=\sum_{i=0}^na_i=\sum_{i=0}^n(a_0+nd)=(n+1)a_0+d\sum_{i=0}^ni\]
于是这个求和就归结为对自然数序列$0,1,2,\ldots,n$进行求和,高斯曾经口算出$1+2+3+\cdots+100=5050$,他的计算方法是,将1与100相加得101,2与99相加得101,如此这般,最后50与51相加得101,于是$1+2+\cdots+100=50 \times 101=5050$,把这个方法略加改造便得出如下的计算$1+2+\cdots+n$的方法,为了避免$n$的奇偶性带来分组分不完从而还需要分奇偶讨论的问题,再拿一个式子$n+(n-1)+\cdots+2+1$与$1+2+\cdots+(n-1)+n$进行相加,则这两个式子对应位置上的两数之和都是$n+1$,所以最终得公式
\[ 1+2+\cdots+n = \frac{1}{2}n(n+1) \]
不必怀疑右端会有出现小数的可能,因为作为连续的两个自然数$n$和$n+1$中必定一奇一偶,所以右端永不会为小数。

把这公式应用到上面的式子便得
\begin{theorem}
  在公差为$d$的等差数列$a_n$中,有
  \[ S_n=\sum_{i=0}^na_i=(n+1)a_0+\frac{1}{2}n(n+1)d \]
  其中$n$可以是任意整数(包括负整数)。
\end{theorem}

上面求$1+2+\cdots+n$的方法称为\emph{倒序相加法},当然也可以直接把这方法应用到$S_n=a_0+a_1+\cdots+a_n$上,同样可得出上面的结果.

\begin{definition}
  如果一数列$\ldots,a_{-2},a_{-1},a_0,a_1,a_2,\ldots$中,任意一项与前一项的比值都是同一非零常数,即存在非零常数$q$,使得对于任意整数$n$都有$a_n / a_{n-1} = q$成立,则称此数列是 \emph{等比数列},而比值常数$q$称为它的 \emph{公比}.
\end{definition}

对于等比数列,如果它的各项都是正的,取对数即可变身为一个等差数列。对于所有的等比数列,同样可以得出类似于等差数列的结论来,即是

\begin{theorem}
对于公比为$q$的等比数列$\cdots,a_{-2},a_{-1},a_0,a_1,a_2,\cdots$,对任意整数$n$和任意整数$m$都成立
\[ \frac{a_n}{a_m}=q^{n-m} \]
\end{theorem}

并由此得出通项
\begin{theorem}
  公比为$q$的等比数列$\ldots,a_{-2},a_{-1},a_0,a_1,a_2,\ldots$的通项可表为
 \[ a_n=a_0q^n \]
\end{theorem}

关于等比数列的求和
\[ S_n = a_0+a_1+\cdots+a_n = a_0(1+q+\cdots+q^{n}) \]
这就归结为对$1+q+q^2+\cdots+q^{n}$求值,

由数学归纳法可得出下面的乘法公式,它也是平方差公式和立方差公式的自然推广(见\autoref{example:a-power-n-b-power-n-factoring})
\begin{equation}
  \label{eq:a-power-n-substract-b-power-n}
 a^n-b^n = (a-b)(a^{n-1}+a^{n-2}b+\cdots+ab^{n-2}+b^{n-1}) 
\end{equation}
式中$n$是任意正整数,

在式中把$n$换成$n+1$并取$a=1,b=q$便可得出,对于任意正整数$n$,成立
\[ 1+q+\cdots+q^{n} = \frac{1-q^{n+1}}{1-q} \]
若$n$为负整数,则上式中以$1/q$代$q$,便知上式仍然成立,于是对于等比数列就有
\begin{theorem}
  在以$q(q \neq 1)$为公比的等比数列$\ldots,a_{-2},a_{-1},a_0,a_1,a_2,\ldots$中,对于任意整数$n$都有
  \[ S_n = \sum_{i=0}^na_i = \frac{a_0(1-q^{n+1})}{1-q} \]
  而在$q=1$时,$S_n=(n+1)a_0$.
\end{theorem}

还可以以另外一种方式得出这公式,因为
\[ S_n = a_0+a_1+\cdots+a_n \]
两端同乘以公比$q$,便可得出
\[ qS_n = q(a_0+a_1+\cdots+a_n) = a_1+a_2+\cdots+a_n+a_{n+1} \]
于是
\[ (1-q)S_n = a_0-a_{n+1} = a_0(1-q^{n+1}) \]
所以得出
\[ S_n = \frac{a_0-a_{n+1}}{1-q} = \frac{a_0(1-q^{n+1})}{1-q} \]
当然这过程中要求$q \neq 1$,而对于$q=1$的情况,$S_n=(n+1)a_0$.

这个方法称为 \emph{错位相减法},下面这个例子提示了这个方法的一些用途。

\begin{example}[错位相减法的一个用途]
  考虑对数列$a_n = n q^n(n=1,2,3,\ldots)$进行求和,这里要求$q \neq 1$,否则就变成早已解决过的问题了。同样有
  \[ S_n = a_1+a_2+\cdots+a_n = q+2q^2+3q^3 + \cdots + (n-1)q^{n-1} + nq^n \]
  两端同乘以$q$得
  \[ qS_n = q^2+2q^3+3q^4 + \cdots + (n-1)q^n + nq^{n+1} \]
  把以上两式相减,得
  \[ (1-q)S_n = (q + q^2 + \cdots + q^n) - n q^{n+1} \]
  显然上式右端括号中的部分是等比数列的求和,于是$S_n$便可以求出。

  接下来再考虑数列$a_n=n^mq^n(n=1,2,3,\ldots)$,其中$m$是个固定的正整数,记
  \[ S_m(n) = \sum_{i=1}^na_i = \sum_{i=1}^n n^m q^n = q + 2^mq^2 + \cdots + (n-1)^mq^{n-1} + n^m q^m \]
  进行同样的过程有
  \[ qS_m(n) = q^2 + 2^mq^3 + \cdots + (n-1)^m q^n + n^m q^{n+1} \]
  两式相减
  \[ (1-q)S_m(n) = q +(2^m-1)q^2 + (3^m-2^m)q^3 + \cdots + (n^m-(n-1)^m)q^m - n^mq^{n+1}  \]
  而利用下面的公式(它可以由二项式定理得出)
  \[ (1+k)^n-k^n = 1+nk+C_n^2k^2 + \cdots + C_n^{n-1}k^{n-1} \]
  上式可化为
  \begin{eqnarray*}
    (1-q)S_m(n) & = & \sum_{i=1}^n(i^m-(i-1)^m)q^i - n^mq^{n+1} \\
                & = & \sum_{i=0}^{n-1}((1+i)^m-i^m)q^{i+1} - n^mq^{n+1} \\
                & = & \sum_{i=0}^{n-1}\sum_{j=0}^{m-1}C_m^ji^jq^{i+1} - n^mq^{n+1} \\
                & = & \sum_{j=0}^{m-1}C_m^j\sum_{i=0}^{n-1}i^jq^{i+1} - n^mq^{n+1} \\
    & = & q\sum_{j=0}^{m-1}C_m^jS_j(n-1) - n^mq^{n+1}
  \end{eqnarray*}
  这意味着$S_m(n)$可以用$S_0(n-1),S_1(n-1),\ldots,S_{m-1}(n-1)$来表示,即具有递推性,而$S_0(n-1)$显然就是等比数列求和,于是$S_m(n)$便可以求出来。

  进一步,便可以求形如$a_n=p(n)q^n$这样的数列的和,其中$p(n)$是关于$n$的多项式,只是这过程将越来越繁琐,如此机械化的计算方法,由计算机程序来进行是再合适不过了。
\end{example}

\begin{example}[自然数的幂和]
  \label{example:sum-of-power-of-integer}
  在上面得出了公式
  \[ 1+2+\cdots+n = \frac{1}{2}n(n+1) \]
  现在就来讨论下一般的$S_m(n)=\sum_{i=1}^ni^m$的求和公式,仍然由
  \[ (1+k)^m-k^m = 1 + mk + C_m^2k^2 + \cdots + C_m^{m-1}k^{m-1} \]
  对$k=1,2,\ldots,n$进行累加,便得
  \[ (1+n)^m-1 = S_0(n)+C_m^1S_1(n)+C_m^2S_2(n) + \cdots C_m^{m-1}S_{m-1}(n) \]
  显然这便是联系着诸$S_m(n)$的递推关系,把它写成更美观的形式
  \[ \sum_{i=0}^mC_m^iS_i(n) = (1+n)^{m+1}-1 \]
  且有初始公式
  \[ S_0(n) = n \]
  由此出发便能求出任何$S_m(n)$的表达式来,比如说取$m=2$便得
  \[ S_0(n)+2S_1(n)+S_2(n) = (1+n)^3-1 \]
  于是得出
  \[ S_2(n) = 1^2+2^2+\cdots+n^2 = \frac{1}{6}n(n+1)(2n+1) \]
  同样不必怀疑右端的整数性,因为$n$与$n+1$作为相邻两个正整数,必然有一个偶数,故$2 \mid n(n+1)$,还需证明$3 \mid n(n+1)(2n+1)$,若是$n$与$n+1$中有3的倍数,则自然不消说,若是$n$与$n+1$都不是3的倍数,则它俩被3除所得余数必然一个是1,另一个是2,于是$2n+1=n+(n+1)$便必然是3的倍数,于是2和3都能整除$n(n+1)(2n+1)$且2与3互素,所以6也能整除它。

  同样再取$m=3$,便可得出
  \[ S_3(n)= 1^3+2^3+\cdots+n^3 = \frac{1}{4}n^2(n+1)^2 \]
  一般的可以知道,$S_m(n)$是关于$n$的$m+1$次多项式。
\end{example}

\subsection{差分与高阶等差数列}
\label{sec:difference-and-high-level-common-difference-sequence}

\begin{definition}
  对于无穷数列$\ldots,a_{-2},a_{-1},a_0,a_1,a_2,\ldots$,定义它的 \emph{一阶差分数列} 为
  \[ \Delta a_n = a_n-a_{n-1} \]
  进一步递归的定义它的$m$阶差分数列是
  \[ \Delta^ma_n = \Delta^{m-1}a_n - \Delta^{m-1}a_{n-1} \]
  即$m$阶差分是$m-1$阶差分的差分,这称为 \emph{高阶差分},为了方便,可以把数列本身看作它自己的 \emph{零阶差分数列}.
\end{definition}

差分将一个数列变换为另一个数列,这与平方运算把一个数变成另一个数,求导运算把一个函数变成另一个函数没什么差别,所以差分是在数列这种数学对象上的一种运算。

为了讨论的方便,用$a_n^{(m)}$代表数列$a_n$的$m$阶差分数列,并规定$a_n^{(0)}=a_n$,上标加了括号,以与幂相区别,在这定义就有
\[ a_n^{(m)} = a_n^{(m-1)} - a_{n-1}^{(m-1)} \]

接下来有高阶等差数列的定义
\begin{definition}
  如果一个数列$a_n$的$m$阶差分数列是等差数列,则该数列称为 \emph{$m$阶等差数列}.
\end{definition}

显然等差数列本身即是零阶等差数列,同时,如果一个数列是$m$阶等差数列,则它也必然是$m+1$阶等差数列,$m+2$阶等差数列,等等。

易见
\begin{theorem}
  数列$a_n$是$m$阶等差数列的充分必要条件是$\Delta a_n$是$m-1$阶等差数列.
\end{theorem}

假定$a_n$是一个$m$阶等差数列,那么存在常数$d$,使得对任意整数$n$成立下式
\[ a_n^{(m)} - a_{n-1}^{(m)} = d \]
把式中的$m$阶差分数列用$m-1$阶差分的差分来替代,就得到
\[ a_n^{(m-1)}-2a_{n-1}^{(m-1)}+a_{n-2}^{(m-1)} = d \]
同样的手段再施行两次,得
\begin{eqnarray*}
 a_n^{(m-2)}-3a_{n-1}^{(m-2)}+3a_{n-2}^{(m-2)}-a_{n-3}^{(m-2)} & = & d  \\
 a_n^{(m-3)}-4a_{n-1}^{(m-3)}+6a_{n-2}^{(m-3)}-4a_{n-3}^{(m-3)}+a_{n-4}^{(m-3)} & = & d 
\end{eqnarray*}
易见各项式的系数正好便是二项式系数并交错的带上正负号,并且$m-k$阶差分数列的公式中便含有相邻$k+2$项,照此规律,当差分的阶数降为零时,公式成为
\[ C_{m+1}^0a_n-C_{m+1}^1a_{n-1}+\cdots+(-1)^{m+1}C_{m+1}^{m+1}a_{n-m-1} = d \]
这便是$m$阶等差数列的一个递推公式,但是公式中带有未知常数$d$,现在想办法去掉它,因为对于$m$阶等差数列而言,它的$m$阶差分数列成为等差数列,那么它的$m+1$阶差分数列就成为一个常数数列,也就是公差为零的等差数列,因此按照上面的公式,原来数列的递推公式可以表为
\[ C_{m+2}^0a_n-C_{m+2}^1a_{n-1}+\cdots+(-1)^{m+2}C_{m+2}^{m+2}a_{n-m-2} = 0 \]
于是得到以下重要结果
\begin{theorem}
  \label{theorem:recursive-for-high-level-common-difference-sequence}
  $m-2$阶等差数列的递推公式是
\[ C_m^0a_n-C_m^1a_{n-1}+\cdots+(-1)^mC_m^ma_{n-m} = 0 \]
\end{theorem}
这只要把上面的推导,改用数学归纳法写出来就可以证明它。

这结果对于等差数列即零阶等差数列而言就成为$a_n-2a_{n-1}+a_{n-2}=0$.

现在讨论一下高阶等差数列的通项问题,先考虑一阶等差数列$a_n$,按定义,它的一阶差分数列是等差数列,于是有
\[ a_n - a_{n-1} = rn+s \]
于是对于整数$n$(包括负整数),就有
\begin{eqnarray*}
  a_n & = & a_0 + \sum_{i=1}^n(a_i-a_{i-1}) \\
      & = &  a_0 + \sum_{i=1}^n(ri+s) \\
      & = &  a_0 + ns + r\sum_{i=1}^n i \\
  & = & a_0 + ns + \frac{1}{2}rn(n+1)
\end{eqnarray*}
这表明它的通项是关于$n$的二次多项式,再重复同样的过程并利用自然数的幂和公式(\autoref{example:sum-of-power-of-integer})便可得出二阶等差数列的通项是关于$n$的三次多项式,如此等等,于是得出如下重要结果
\begin{theorem}
  \label{theorem:common-formular-for-high-level-common-difference-sequence}
  无穷数列$\ldots,a_{-2},a_{-1},a_0,a_1$是$m$阶等差数列的充分必要条件是,它的通项能表成关于下标的$m+1$次多项式.
\end{theorem}

这结果表明,高阶等差数列本质上就是多项式数列。

\begin{proof}[证明]
  对$m$施行数学归纳法,$m=0$的情况是显然的,假设必要性对于小于$m$的正整数都成立,那么对于一个$m$阶等差数列$a_n$,它的一阶差分数列是$m-1$阶等差数列,因而通项为关于下标的$m$次多项式,于是
  \[ a_n-a_{n-1} = \sum_{j=0}^m b_jn^j \]
  累加得
  \begin{eqnarray*}
    a_n & = & a_0 + \sum_{i=0}^n(a_i-a_{i-1}) \\
        & = & a_0 + \sum_{i=0}^n\sum_{j=0}^mb_ji^j \\
    & = & a_0 + \sum_{j=0}^mb_j\sum_{i=0}^ni^j \\
  \end{eqnarray*}
  在\autoref{example:sum-of-power-of-integer}中,我们就已经知道$\sum_{i=1}^ni^j$是$j+1$次多项式,于是$a_n$便是关于$n$的$m+1$次多项式,所以必要性成立.

  再证充分性,同样施行归纳法,如果数列的通项是关于下标的一次多项式,显然它必然是(零阶)等差数列,即$m=0$时充分性成立,假若充分性对于小于$m$的正整数都成立,现在设数列$a_n$的通项是关于$n$的$m+1$次多项式,即
  \[ a_n = \sum_{i=0}^{m+1}b_in^i \]
  则它的差分数列
  \[ a_{n+1}-a_n = \sum_{i=0}^{m+1}b_i((1+n)^i-n^i) \]
  显然,$n^{m+1}$被抵销,这成为一个次数低于$m+1$的多项式,按归纳假设,这差分数列必然是某一阶的等差数列,且阶数小于$m$,于是原来的数列$a_n$也是某一阶的等差数列,且阶数小于等于$m$,自然也是$m$阶等差数列.
\end{proof}

\subsection{线性递推数列}
\label{subsec:linear-recurrence-sequence}

本节讨论常系数线性递推数列的通项求法问题\footnote{线性递推数列的通项求法这部分内容,主要参考了文献\cite{olympic-math}.},这个都是有固定结论的内容,本文只是粗略转叙一番而已。

所谓常系数线性递推数列,是指它的递推公式形如$\sum_{k=1}^n \lambda _k a_k = c$的数列,例如斐波那契数列$a_{n+2}=a_{n+1}+a_n$。

先看最简单的一种,递推式为$a_{n+1}=pa_n+q$的数列,当$p=1$时它成为等差数列,当$q=0$则成为等比数列,所以此处限定$p \neq 1, q \neq 0$。

要求它的通项,只要在它两端同时除以$p^{n+1}$,就有
\[ \frac{a_{n+1}}{p^{n+1}} = \frac{a_n}{p^n} + \frac{q}{p^{n+1}} \]
因此有
\begin{eqnarray*}
\frac{a_n}{p^n} & = & \frac{a_1}{p} + \sum_{k=2}^{n}\left( \frac{a_k}{p^k} - \frac{a_{k-1}}{p^{k-1}} \right) \\
& = & \frac{a_1}{p} + \sum_{k=2}^{n}\frac{q}{p^k}
\end{eqnarray*}
剩下的就是对一个等比数列进行求和了。

另外一种方法比较巧妙,假定存在一个实数$\lambda$,使得$a_{n+1}+\lambda=p(a_n+\lambda)$,展开与原递推式比较即得$\lambda=\frac{q}{p-1}$,于是数列$a_n-\lambda$就成为一个等比数列了。

现在看二阶的情形,每一项需要它前面两项才能确定:$a_{n+2}=pa_{n+1}+qa_n$,假想有两个实数$r$和$s$能够使得
\begin{equation}
  \label{eq:two-level-linear-recurrence-sequence-1}
a_{n+2}-ra_{n+1}=s(a_{n+1}-ra_n)
\end{equation}
展开后与原递推式比较可得
\begin{align*}
  r+s  =  p \\
  rs  =  -q
\end{align*}
因此$r$和$s$是方程$x^2=px+q$的两个根,在复数范围内,它必有两个解(可以相等,重根按重数计算),于是数列$a_{n+1}-ra_{n}$成为等比数列,求出它的通项$a_{n+1}-ra_n=f(n)$后,只要两端同时除以$r^{n+1}$即可求出$a_n$的通项,其结果如下

如果方程有两个等根$x=r$,那么
\[ a_n = \left( \frac{a_2-ra_1}{r^2} - \frac{a_2-2ra_1}{r^2} \right)r^n \]
如果方程有两个不相等的根$x_1=r,x_2=s$,那么
\[ a_n = \frac{a_2-sa_1}{r(r-s)} r^n - \frac{a_2-ra_1}{s(r-s)} s^n \]

如果这两个根不相等,则还有另一种求法,因为$r$和$s$都是方程$x^2=px+q$的根,因此既然有\ref{eq:two-level-linear-recurrence-sequence-1}成立,也就必然有
\begin{equation}
  \label{eq:two-level-linear-recurrence-sequence-2}
a_{n+2}-sa_{n+1}=r(a_{n+1}-sa_n)
\end{equation}
成立,递推下去,便有
\begin{align*}
  a_{n}-ra_{n-1}  =  s^{n-2}(a_2-ra_1) \\
  a_{n}-sa_{n-1}  =  r^{n-2}(a_2-sa_1) 
\end{align*}
从中解出$a_n$来:
\[ a_n=\frac{(a_2-sa_1)r^{n-1}-(a_2-ra_1)s^{n-1}}{r-s} \]
易见这公式具有如下形式
\[ a_n=c_1r^n+c_2s^n \]
其中$c_1$和$c_2$是常数,这常数跟$a_1$、$a_2$以及$r$和$s$有关,跟$r$和$s$有关实际上就是跟$p$与$q$有关,可以直接将$a_1$和$a_2$的值代入上式中定出$c_1$和$c_2$,这只需求解一个二元一次方程组就可以了。

而对于有两个等根的情形,通项具有形式
\[ a_n = (c_1+c_2n)r^n \]
的情形,同样可以利用待定系数法求出$c_1,c_2$.

更一般的情形是
\begin{theorem}
对于线性递推数列$\sum_{k=1}^n\lambda_k a_{p+k}=c$,称方程$\sum_{k=1}^n\lambda _k x^k = c$为它的特征方程,在复数范围内这个特征方程必有$n$个根(重根按重数计算),假定这些根是 $x_i(i=1,2,\cdots,m, m \leq n)$,相应根的重数是$r_i(i=1,2,\cdots,m, \sum_{i=1}^mr_i=n)$,则它的通项是:
\[ a_n=\sum_{i=1}^mP_{r_i-1}(n)x_i^n \]
上式中$P_{r_i-1}(n)$表示一个关于$n$的次数是$r_i-1$的多项式,如果哪个根是单重根,则它的系数是常数。
\end{theorem}

此定理便是说,线性递推数列,其实就是多项式作系数的指数的组合,以后在微分方程中还会看到,常系数线性微分方程的解也是指数函数的组合,这两个结果表明,差分与微分存在某种类似,差分可以视为微分的离散化。

\begin{example}
作为一个例子,现在来求斐波那契数列的通项,递推公式为$a_{n+2}=a_{n+1}+a_n$,特征方程是$x^2=x+1$,其两个根是$x_{1,2}=\frac{1}{2}(1 \pm \sqrt{5})$,于是通项应为:
\[ a_n=\alpha _1 x_1^n+ \alpha _2 x_2^n \]
将$a_1=1$和$a_2=1$带入求出两个系数,最后得:
\[ a_n= \frac{1}{\sqrt{5}}\left[ \left( \frac{1+\sqrt{5}}{2} \right)^n - \left( \frac{1-\sqrt{5}}{2} \right)^n \right] \]
令人惊讶的是一个所有项都是正整数的数列,其通项居然出现了无理数,事实上,利用二项定理可以证明,这个表达式将永远是正整数。
\end{example}

\begin{example}
  讨论数列$a_1=1,a_2=1$,而以后的项由$a_{n+2}=a_{n+1}-a_n$决定,其特征方程是$x^2=x-1$,方程在复数范围内有两个根
  \[ \alpha = \frac{1+\sqrt{3}i}{2}, \  \beta = \frac{-1+\sqrt{3}i}{2} \]
  通项则有形式$a_n=c_1 \alpha^n + c_2 \beta^n$的形式,将$a_1$和$a_2$的值代入通项定出$c_1$和$c_2$后得
  \[ a_n = \frac{1}{\sqrt{3}i} \left( \frac{1+\sqrt{3}i}{2} \right)^{n} - \frac{1}{\sqrt{3}i} \left( \frac{-1+\sqrt{3}i}{2} \right)^n \]
 这时的通项,就不得不借助复数来表达了,实际上这还是一个周期数列。
\end{example}

\begin{example}
  我们已经知道,一个$m-2$阶等差数列的递推公式是
\[ C_m^0a_n-C_m^1a_{n-1}+\cdots+(-1)^mC_m^ma_{n-m} = 0 \]
这显然是一个线性递推数列,其特征方程是
\[ \sum_{i=0}^m(-1)^iC_m^ix^i = 0 \]
也就是$(1-x)^m = 0$,这方程只有一个$m$重根$x=1$,因而通项可表为$a_n=p(n) \cdot 1^n$,其中$p(n)$是一个$m-1$次多项式,这样就再次证明了\autoref{theorem:common-formular-for-high-level-common-difference-sequence}.
\end{example}



\subsection{分式型递推数列}
\label{sec:general-formular-of-frac-recursive-series}

\subsection{递推方法的应用}
\label{sec:application-of-recursive-method}

有些问题本身并不是数列问题,但因为问题与自然数有关,递推方法往往能发挥作用,这一节举几个例子。

\begin{example}[伯努利信封问题]
  \label{example:bernoulli-envelope-solved-by-recursive-method}
  伯努利信封问题早在介绍容斥原理时就解决过(见\autoref{example:bernoulli-envelope-solved-by-inc-exclu-principle}),此处介绍运用数列递推方法的解决方式,有两个方法,其一是大数学家欧拉所提供的,思维精巧,其二是我自己的解法。

  简单复述一下此问题,有相同数目的一一配对的信件和信封,把这些信件装进这些信封中,问没有任何一封信件装进正确的信封的装法总数是多少。

  
  欧拉对此问题有一个借助数列递推的解法,记信封数目为$n$时的结果为$a_n$,那么$a_1=0$,来看下这个数列的递推情况,在有$n+1$个信封时,考虑编号为1的那封信件,除1号信封外它有$n$个信封可以装入,假定它装入的信封编号是$r$,那再考虑编号为$r$的信件,它此时有两个选择,一是它可以装入1号信封,这时其它的$n-1$个信件的装法是$a_{n-1}$,它的另一个选择是不装入1号信封,这时由于1号信封等同于$r$号信封,所以除1号信件以外的$n$封信件有$a_n$种装法,于是得到该数列的递推公式为:
  \begin{equation*}
    a_{n+1}=n(a_n+a_{n-1})
  \end{equation*}
  两边同除以$(n+1)!$并记$b_n=\frac{a_n}{n!}$可得
  \begin{equation*}
    b_{n+1}=\frac{n}{n+1}b_n+\frac{1}{n+1}b_{n-1}
  \end{equation*}
  即
  \begin{equation*}
    b_{n+1}-b_n=-\frac{1}{n+1}(b_n-b_{n-1})
  \end{equation*}
  所以
  \begin{equation*}
    b_n-b_{n-1}= \sum_{i=2}^{n}(b_i-b_{i-1})= \sum_{i=2}^n (-1)^{i}\frac{1}{i!}
  \end{equation*}
  于是
  \begin{equation*}
    a_n= n!b_n = n!(b_1 + \sum_{i=2}^n(b_i-b_{i-1})) = n! \sum_{i=2}^n(-1)^{i}\frac{1}{i!}
  \end{equation*}
  同样在$0!=1$的约定下即有
  \begin{equation*}
    a_n = n!\sum_{i=0}^n(-1)^{i}\frac{1}{i!}
  \end{equation*}

  以下介绍另一个解法,这个解法是我在高中时期搞出来的,那时我还没听说过伯努利信封问题,当时考虑的问题是:有编号为$1,2,\ldots,n$的$n$个人和同样编号的$n$个座位,如果每个人都不允许坐与自身编号相同的座位,有多少种坐法?我当时称之为错位排列问题,原稿早已遗失(也太啰嗦,长达三页,当时花费了好多天,写出来后自己不会word,花了十元钱请打印店打出来,成天自我欣赏,这大概代表了我高中时期在数学上的最高成就),此处是根据思路重新写的。

考虑编号为1的人,假如他坐在编号为$r_1$的座位上,又看编号为$r_1$的人,他可以坐在1号座位上,也可以坐在$r_2$号座位上,如果是坐在1号座位上,那么1号和$r_1$号这两个人由于交叉坐,与其他人无涉,如果是坐在$r_2$号座位上,则再继续看$r_2$号人,依此顺着链条$1\rightarrow r_1 \rightarrow r_2 \rightarrow \cdots$,这样下去的结果是,从1号人开始,能够找到一个环,使得该环内的人和座位与环外的人和座位无涉,可以独立开来,如果找不到这样的环,那无非是所有人一起构成了一个整体环,即是无法分割。这样的环可能不止一个,但我们仅考虑包含1号人的那个环。

假定$n$个人的错位排列数为$a_n$,设包含1号人的环共含有$m(2\leqslant m \leqslant n-2)$个人,那么只要从除1号之外的人中选出$m-1$个人排在这个座位链上,因此座法是$(m-1)!C_{n-1}^{m-1}=\dfrac{(n-1)!}{(n-m)!}$,而剩下的$n-m$个人又组成了一个较小的错位排列,因此其坐法数是$a_{n-m}$,此外,当$m=n$时,所有人构成一个环,只要把他们按照座位链排成一排即可,所以此时的坐法是$(n-1)!$,于是总共的坐法是:
\begin{equation*}
  a_n=\sum_{m=2}^{n-2}\frac{(n-1)!}{(n-m)!}a_{n-m}+(n-1)!=(n-1)!(\sum_{m=2}^{n-2}\frac{a_{n-m}}{(n-m)!}+1)
\end{equation*}
将式中的求和顺序倒过来,就有
\begin{equation}
  \label{eq:requsive-equation-derangement}
  a_n=(n-1)!(\sum_{m=2}^{n-2}\frac{a_{m}}{m!}+1)
\end{equation}
此即其作为数列的递推公式,自然的,$a_1=0,a_2=1$。

下面来求它的通项公式,上式可以变形为:
\begin{equation}
  \label{recursive-equation-derangement-2}
  \frac{a_n}{n!}=\frac{1}{n}(1+\sum_{m=2}^{n-2}\frac{a_m}{m!})
\end{equation}
只要记$b_n=\dfrac{a_n}{n!}$,则有$b_1=0$和
\begin{equation}
  \label{eq:recursive-equation-derangement-simple}
  b_n=\frac{1}{n}(1+\sum_{m=2}^{n-2}b_m)
\end{equation}
于是有
\begin{equation}
  \label{eq:recursive-equation-derangement-simple-diff}
  nb_n-(n-1)b_{n-1}=b_{n-2}
\end{equation}
也就是
\begin{equation}
  \label{eq:recursive-equation-derangement-simple-diff2}
  b_n-b_{n-1}=-\frac{1}{n}(b_{n-1}-b_{n-2})
\end{equation}
而$b_2-b_1=\dfrac{1}{2}$,所以
\begin{equation}
  \label{eq:recursive-equation-derangement-bn}
  b_n-b_{n-1}=(-1)^{n-2}\frac{1}{n!}=(-1)^n\frac{1}{n!}
\end{equation}
于是
\begin{equation}
  \label{eq:derangement-bn}
  b_n=b_1+\sum_{m=2}^{n}(b_m-b_{m-1})=\sum_{m=2}^n(-1)^m\frac{1}{m!}
\end{equation}
所以最终
\begin{equation}
  \label{eq:derangement-an}
  a_n=n!b_n=n!\sum_{m=2}^n(-1)^m\frac{1}{m!}
\end{equation}
在$0!=1$的约定下,也可以把求和指标从零开始
\begin{equation}
  \label{eq:derangement-an2}
  a_n=n!\sum_{m=0}^n(-1)^m\frac{1}{m!}
\end{equation}
这就是最终的结果。

\end{example}

\subsection{题选}
\label{sec:exercise-for-number-series}



\begin{exercise}
  已知各项都是正实数的数列$x_n$对一切正整数$n$都成立$x_n+\frac{1}{x_{n+1}}<2$,求证该数列所有项都满足$x_n<1$.
\end{exercise}
\begin{proof}[解答]
  如果用上极限理论,则可以很容易的得出它单调增加并以1为极限,结论不证自明,所以这里主要讨论的是初等证明。

因为
$$
x_n+\frac{1}{x_{n+1}}<2 \leqslant 
x_{n+1}+\frac{1}{x_{n+1}}
$$
所以$x_n<x_{n+1}$,即该数列单调增加。

又显然$x_n<2$,所以
$$
2>x_n+\frac{1}{x_{n+1}}>x_n+\frac{1}{2}
$$
于是$x_n<2-\frac{1}{2}$,我们得到一个更加好的上限,重复这个过程,我们由$x_n<y_m$就可以得到
$$
x_n<2-\frac{1}{y_m}
$$
所以我们作数列$y_m$,它由$y_1=2$和
$$
y_{m+1}=2-\frac{1}{y_m}
$$
来确定。

数列$y_m$的每一项都大过数列$x_n$的全部项,所以它的下标特意用$m$而不是$n$来表示,以示不相关。

现在来求$y_m$的通项公式,由于
$$
\frac{1}{y_{m+1}-1}=1+\frac{1}{y_{m}-1}
$$
因此数列$\frac{1}{y_m-1}$是等差数列,它的通项为$y_m=1+\frac{1}{m}$\footnote{这是分式型递推数列求通项的不动点解法。},于是$x_n<y_m$对一切正整数$n$和$m$都成立,所以必定有$x_n\leqslant 1$(用反证法),而$x_n$的单调性则保证了等号是不能取的。

下面关于$y_m$再给个不求通项的玩法\footnote{其实这是高数的玩法,就差提到确界二字了。},$y_m>1$这一点根据数学归纳法是明显成立的。下面证明它可以任意接近1,也就是要证明,对于无论多么小的正实数$\delta$,总存在$y_m$中的某一项$y_M$,使得$y_M<1+\delta$。
采用反证法,假定存在某个正实数$\delta$,使得$y_m$中的所有项都满足$y_m\geqslant 1+\delta$,则
$$
y_{m+1}-1=\frac{1}{y_m}(y_m-1)\leqslant 
\frac{1}{1+\delta}(y_m-1)
$$
于是
$$
y_m-1\leqslant \frac{1}{(1+\delta)^m}
$$
显然与假设矛盾,故得证。
\end{proof}

\begin{exercise}
  数列$a_n$满足:$a_1=\frac{1}{2}$,$a_{n+1}=\frac{na_n+a_n^2}{n+1}$,
  \begin{enumerate}
  \item 求证该数列是递减的。
  \item 求证 $a_n < \frac{7}{4n}$
  \end{enumerate}
\end{exercise}
\begin{proof}[解答]
  根据数列归纳法易知 $0<a_n<1$,所以
  \begin{equation*}
    a_{n+1}=\frac{na_n+a_n^2}{n+1} < \frac{na_n+a_n}{n+1} = a_n
  \end{equation*}
  所以数列递减,第二问,只要证明$n>3$时有如下更强的不等式即可(使用数学归纳法,过程略去)
  \begin{equation*}
    a_n \leqslant \frac{7}{4n}-\frac{4}{n^2}
  \end{equation*}
\end{proof}

\begin{exercise}
  记 $I_n=1-\frac{1}{2}+\frac{1}{3}-\frac{1}{4}+\cdots+\frac{1}{2n-1}-\frac{1}{2n}$,求证,在正整数$n\geqslant 100$时,有$0.68<I_{n}<0.7$.
\end{exercise}
\begin{proof}[证明]
  记$J_n=\sum_{k=1}^n(\frac{1}{2k}-\frac{1}{2k+1})$,由于$z_n=\frac{1}{n-1}-\frac{1}{n}=\frac{1}{n(n-1)}$是递减的,并且相邻两项也相差越来越小,所以有不等式$z_{2k}<\frac{1}{2}(z_{2k-1}+z_{2k+1})$,也就是如下的:
\begin{equation*}
  \frac{1}{2k-1} - \frac{1}{2k} < \frac{1}{2} \left[ \left( \frac{1}{2k-2} - \frac{1}{2k-1} \right) + \left( \frac{1}{2k} - \frac{1}{2k+1} \right) \right]
\end{equation*}
对上式左边进行累加,但从$k=3$到$k=n$使用右边放缩,得
\begin{equation*}
  I_n<\frac{1}{2}+\frac{1}{12}+\frac{1}{2} \left[ \left( J_n-\frac{1}{6}-\frac{1}{2n(2n+1} \right) + \left( J_n-\frac{1}{6}-\frac{1}{20} \right) \right]
\end{equation*}
化简
\begin{equation*}
 I_n<J_n+\frac{47}{120} - \frac{1}{4n(2n+1)}
\end{equation*}
利用$I_n+J_n=1-\frac{1}{2n+1}$从上式中换掉$J_n$得
\begin{equation}
  \label{eq:sign-sum-reciprocal-positive-integer-max}
  I_n<\frac{167}{240}-\frac{1}{2(2n+1)}-\frac{1}{8n(2n+1)}<\frac{167}{240}<\frac{168}{240}=0.7
\end{equation}
于是不等式的右边得证,接下来考虑左边不等式,同样因为$z_n$是递减的,有不等式$z_{2k}>\frac{1}{2}(z_{2k}+z_{2k+1})$,也就是
\begin{equation*}
  \frac{1}{2k-1} - \frac{1}{2k} > \frac{1}{2} \left[ \left( \frac{1}{2k-1} - \frac{1}{2k} \right) + \left( \frac{1}{2k} - \frac{1}{2k+1} \right) \right]
\end{equation*}
对左边进行累加,在$k \geqslant 3$ 时使用右边放缩,得到
\begin{equation*}
  I_n > \frac{1}{2} + \frac{1}{12} + \frac{1}{2} \left( \frac{1}{5} - \frac{1}{2n+1} \right)
\end{equation*}
也就是
\begin{equation}
  \label{eq:sign-sum-reciprocal-positive-integer-min}
 I_n > \frac{41}{60} -\frac{1}{2(2n+1)} 
\end{equation}
在$n \geqslant 100$时,有
\begin{equation*}
  I_{n} \geqslant I_{100} > \frac{41}{60} - \frac{1}{402} = 0.680845771... > 0.68
\end{equation*}
所以不等式左边得证.

其实证明左边所用的放缩是比较松的,实际上因为$\lim_{n\to\infty}I_n=\ln{2}=0.693147...$,所以左边不等式的放缩余地较大,所以这样的放缩也能达到要求,现在来尝试使用更强的放缩,看看能得到一个什么样的结果。

对于$z_{n}$,不等式$z_n>2z_{n+1}-z_{n+2}$将是一个更强的放缩,所以我们有$z_{2k}>2z_{2k+1}-z_{2k+2}$,也就是下面的不等式
\begin{equation*}
  \frac{1}{2k-1}-\frac{1}{2k} > 2 \left( \frac{1}{2k}-\frac{1}{2k+1} \right) - \left( \frac{1}{2k+1}-\frac{1}{2k+2} \right)
\end{equation*}
对上式左边进行累加,但只在$k\geqslant 3$时使用右边放缩,即得
\begin{equation*}
  I_n > \frac{1}{2}+\frac{1}{12} + 2 \left( J_n-\frac{1}{6}-\frac{1}{20} \right) - \left( I_n-\frac{1}{2}-\frac{1}{12}-\frac{1}{30}+\frac{1}{(2n+1)(2n+2)} \right)
\end{equation*}
将其中的$J_{n}$用$1-I_n-\frac{1}{2n+1}$替换掉,即得
\begin{equation*}
  I_n>\frac{83}{120}-\frac{1}{2(2n+1)}-\frac{1}{4(2n+1)(2n+2)}
\end{equation*}
因此在$n \geqslant 100$时,便有
\begin{equation*}
  I_n \geqslant I_{100} > \frac{83}{120}-\frac{1}{2(2\times 100+1)}-\frac{1}{4(2\times 100+1)(2 \times 100 + 2)} = 0.6891729471.....
\end{equation*}
这个值已经非常接近$0.69$了。
\end{proof}

\begin{exercise}
  已知数列$\{a_n\}$满足: $a_1=1$,$a_{n+1}=a_n+\frac{1}{a_n^2}$,求证$a_{2015}>18$.
\end{exercise}

\begin{proof}[证明]
  易见这是一个递增的正项数列,在递推式两边同时三次方:
  \begin{equation*}
    a_{n+1}^3=\left( a_n+\frac{1}{a_n^2} \right)^3 = a_n^3+3+\frac{3}{a_n^3}+\frac{1}{a_n^6}>a_n^3+3
  \end{equation*}
  所以$a_{2015}>a_1^3+3\times 2014=6043 > 5832 = 18^3$.

  遗留问题,如果要证明的是 $18.2<a_{2015}<18.3$呢(编程计算知这是成立的)?
\end{proof}

\begin{exercise}
  数列$a_n$满足$a_1=2$,$(n+1)a_{n+1}^2=na_n^2+a_n$,求证
  \[ \sum_{i=2}^n \frac{a_i^2}{i^2}<\frac{9}{5} \]
\end{exercise}

\begin{proof}[证明一]
  由数列归纳法易证$a_n>1$,所以
\[ a_{n+1}^2=\frac{na_n^2+a_n}{n+1} < \frac{na_n^2+a_n^2}{n+1} = a_n^2 \]
于是数列递减,所以当$n>1$时,$a_n < a_1=2$
\[ (n+1)a_{n+1}^2 = na_n^2+a_n < na_n^2+2\]
于是累加下去,就有
\[ na_n^2 < a_1^2+2(n-1)=2(n+1) \]
所以
\[ a_n^2 < 2 \left( 1+\frac{1}{n} \right) \]
于是
\[ \sum_{i=2}^n \frac{a_i^2}{i^2} <2 \left( \sum_{i=2}^n \frac{1}{i^2}+\sum_{i=2}^n \frac{1}{i^3} \right) \]
借用放缩
\[ \frac{1}{i^2}<\frac{1}{(i-1)i}=\frac{1}{i-1}+\frac{1}{i} \]
和
\[ \frac{1}{i^3}<\frac{1}{(i-1)i(i+1)} = \frac{1}{2} \left( \frac{1}{(i-1)i}-\frac{1}{i(i+1)} \right) \]
从$i \geqslant 4$开始放缩,累加即得
\begin{align*}
\sum_{i=2}^n \frac{a_i^2}{i^2} & < 2 \left( \frac{1}{4}+\frac{1}{9}+(\frac{1}{3}-\frac{1}{n})+\frac{1}{8}+\frac{1}{27}+\frac{1}{2}(\frac{1}{12}-\frac{n}{n+1}) \right) \\
& < 2 \left( \frac{1}{4}+\frac{1}{9}+\frac{1}{3}+\frac{1}{8}+\frac{1}{27}+\frac{1}{24} \right) \\
& = 2 \left( \frac{3}{4}+\frac{4}{27} \right)<\frac{9}{5}
\end{align*}
\end{proof}

\begin{proof}[证明二]
 不以要证的不等式为目标,研究下这个数列的性态,因为
\[ a_{n+1}^2=a_n \frac{na_n+1}{n+1} \]
显然$\frac{na_n+1}{n+1}$是$a_n$和1的加权平均,因为$a_n>1$有$\frac{na_n+1}{n+1}<a_n$,所以有
\begin{align*}
a_{n+1} &=\sqrt{a_n\cdot \frac{na_n+1}{n+1}} \\
& <\frac{1}{2} \left( a_n+\frac{na_n+1}{n+1} \right)  \\
& = \frac{2n+1}{2n+2}a_n+\frac{1}{2n+2}
\end{align*}
另一方面,由$\frac{na_n+1}{n+1}<a_n$,所以
\[ a_{n+1}^2=a_n \cdot  \frac{na_n+1}{n+1} > \left( \frac{na_n+1}{n+1} \right)^2 \]
所以
\[ a_{n+1}>\frac{n}{n+1}a_n+\frac{1}{n+1} \]
综合这两个估计,得到
\[ \frac{n}{n+1}a_n+\frac{1}{n+1} < a_{n+1} < \frac{2n+1}{2n+2}a_n+\frac{1}{2n+2} \]
左右都是$a_n$和1的加权平均,只是权重不同,上式改写为
\[ \frac{n}{n+1}(a_n-1) < a_{n+1}-1 < \frac{2n+1}{2n+2} (a_n-1) \]
所以最后就有估计式
\[ 1+\frac{1}{n} < a_n < 1 + \frac{1}{2} \cdot \frac{(2n-1)!!}{(2n)!!} \]
对于后面的双阶乘,由熟知的放缩
\begin{align*}
& \left( \frac{1}{2} \cdot \frac{3}{4} \cdots \frac{2n-1}{2n} \right)^2 \\
={} & \left( \frac{1}{2} \cdot \frac{1}{2} \right) \left(\frac{3}{4} \cdot \frac{3}{4} \right) \cdots \left( \frac{2n-1}{2n} \cdot \frac{2n-1}{2n} \right) \\
<{} & \left( \frac{1}{2} \cdot \frac{2}{3} \right) \left( \frac{3}{4} \cdot \frac{4}{5} \right) \cdots \left( \frac{2n-1}{2n} \cdot \frac{2n}{2n+1} \right) \\
={} & \frac{1}{2n+1}
\end{align*}
所以$a_n$的估计式两端都以1为极限,由夹逼定理,$a_n$极限为1.

而仍由那估计式,可以得出
\[ a_n^2< \left( 1+\frac{1}{2\sqrt{2n+1}} \right)^2 <2+\frac{1}{4} \frac{1}{2n+1} < 2+\frac{1}{8n} \]
由这不等式,仍同证明一中的放缩,同样可证得题目中的不等式。 
\end{proof}

\begin{exercise}\footnote{题目来自悠闲数学娱乐论坛.}
  数列$a_n$满足,$a_1=1$,$na_na_{n+1}=1$,求证: 
  \begin{enumerate}
  \item
    \[ \frac{a_{n+2}}{n} = \frac{a_n}{n+1} \]
  \item
    \[ 2(\sqrt{n+1}-1) \leqslant \frac{1}{2a_3}+\frac{1}{3a_4}+\cdots+\frac{1}{(n+1)a_{n+2}} \leqslant n \]
  \end{enumerate}
\end{exercise}

\begin{proof}[证明]
  首先在$a_{n+1}a_n=\frac{1}{n}$中将$n$替换为$n+1$得$a_{n+2}a_{n}=\frac{1}{n+1}$,这两式相除即得
\[ \frac{a_{n+2}}{n} = \frac{a_n}{n+1} \]
第一问得证。
第二问,通项
\[ \frac{1}{(k+1)a_{k+2}} = \frac{1}{ka_k} = a_{k+1} \]
所以只是要证明下式
\[ 2(\sqrt{n+1}-1) \leqslant a_2+a_3+\cdots+a_{n+1} \leqslant n \]
这只要能证明下面这个估计就能办到($n>1$)
\[ 2(\sqrt{n}-\sqrt{n-1}) \leqslant a_n \leqslant 1 \]
用数学归纳法试了一下,递推存在一点困难,倒是证明更强的放缩容易些($n>1$)
\[ \frac{1}{\sqrt{n-1}} \leqslant a_n \leqslant 1 \]
以下就来数学归纳法吧,计算得$a_2=1$,$a_3=\frac{1}{2}$,都符合这不等式,于是假定$a_n$满足这不等式,来看$a_{n+2}$的情况:
\[ a_{n+2} = \frac{n}{n+1}a_n < a_n \leqslant 1 \]
同时
\[ a_{n+2} = \frac{n}{n+1}a_n \geqslant \frac{n}{n+1} \cdot \frac{1}{\sqrt{n-1}} > \frac{1}{\sqrt{n+1}} \]
于是得证。
说明:这里数归是从$n$到$n+2$而不是$n+1$是因为,试算了前面几项,发现它是一个锯齿数列,也就是偶数项都向1靠近,奇数项都向0靠近,所以就分别考虑两个子列了。
\end{proof}

\begin{exercise}\footnote{来自kuing的悠闲数学娱乐论坛。}
  数列$a_n$满足: $a_1=1/3$,递推公式为$a_{n+1}=a_n+\frac{a_n^2}{n^2}$,求证:
  \[ \frac{n}{2n+1} \leqslant a_n \leqslant \frac{2n-1}{2n+1} \]
\end{exercise}

\begin{proof}[证明]
  这个递推公式比较有意思,在证出这题目后还可以进一步研究$a_1$的取值对数列的收敛性的影响,因为$a_n=n$是符号这递推式的,而题目的结论表明当$a_1$取$1/3$时数列有上界,显然数列又是递增的,所以是收敛的。

  左边不等式是很松的,因为数列是递增的,而
  \[ a_4=\frac{30760}{59049}=0.520923\cdots>\frac{1}{2}>\frac{n}{2n+1} \]
  右端尚未有思路,但是根据程序显示的结果,数列在$a_4$就已经超过0.57,但是直到$a_{100000}$都还没超过0.61,所以此数列增长极其缓慢,从要证的结论来看它有上界,因而有极限,猜测这极限应该小于1,若能证明这一点,则结论便能得到证明。
\end{proof}

\begin{exercise} 
  两个数列$a_n$和$b_n$满足: $a_1=0$, $b_1=\frac{1}{2}$,并有递推关系
  \[ a_{n+1}=\frac{a_n+b_n}{2}, \  b_{n+1}=\sqrt{a_{n+1}b_n} \]
  求证这两个数列有共同的极限,并求出这个极限。
\end{exercise}

题目来源: 悠闲数学娱乐论坛,原帖链接: \url{http://kuing.orzweb.net/viewthread.php?tid=4569&extra=page%3D1},这题目命题人(网友"其妙")给出了它的几何背景,是周长为2的正$2^n$边形的内切圆半径和外接圆半径.

\begin{proof}[证明一]
  (解答于 2017-05-02)
  因为
\[ \frac{a_{n+1}}{b_{n+1}}=\frac{\frac{a_n+b_n}{2}}{\sqrt{a_{n+1}b_n}} = \frac{\frac{a_n+b_n}{2}}{\sqrt{\frac{a_n+b_n}{2}b_n}} = \sqrt{\frac{1+\frac{a_n}{b_n}}{2}} \]
所以令$c_n=a_n / b_n$,就有
\[ c_{n+1}=\sqrt{\frac{1+c_n}{2}} \]
受半角公式
\[ \cos{\frac{\theta}{2}}=\sqrt{\frac{1+\cos{\theta}}{2}} \]
启发,令$c_n=\cos{\theta_n}$,则可取$\theta_{n+1}=\theta_n / 2$,结合$\cos{\theta_1}=c_1=a_1 / b_1=0$可取$\theta_1=\pi / 2$,于是$\theta_n=\pi / 2^n$,所以得到
\[ a_n=\cos\frac{\pi}{2^n}b_n \]
再回到$b_n$的递推式,有
\[ b_{n+1}^2=a_{n+1}b_n=\cos{\frac{\pi}{2^{n+1}}} b_{n+1}b_n\]
所以
\[ b_{n+1}=\cos{\frac{\pi}{2^{n+1}}} b_n \]
于是便不难求得
\[ b_n=b_1 \cos{\frac{\pi}{2^2}}\cos{\frac{\pi}{2^3}}\cdots \cos{\frac{\pi}{2^n}} \]
对这个余弦的连乘积,将它乘以$\sin{\frac{\pi}{2^n}}$再利用正弦的二倍角公式便会发生连锁反应,反应的结果便是这连乘积等于$\left( 2^{n-1}\sin{\frac{\pi}{2^n}} \right)^{-1}$,所以
\[ b_n=\frac{1}{2^n \sin{\frac{\pi}{2^n}}} \]
而
\[ a_n=\cos{\frac{\pi}{2^n}}b_n=\frac{1}{2^n}\cot{\frac{\pi}{2^n}} \]
由熟知的极限$\lim_{x \to 0} \frac{\sin{x}}{x}=1$便知$a_n$和$b_n$有共同的极限$\frac{1}{\pi}$.

补充说明:在令$c_n=\cos{\theta_n}$时尚需证明$|c_n| \leqslant 1$,这利用$c_n$的递推式和初始值$c_1=0$,由数学归纳法便知$0 \leqslant c_n \leqslant 1$,如果初始$c_1>1$,便不能使用余弦了,但却可以使用双曲函数$\cosh{x}=(e^x+e^{-x})/2$,双曲函数有着相同的倍半公式。
\end{proof}

\begin{proof}[证明二]
  (解答于 2012-05-18)
  首先在$a_n$的递推式两边取极限即知它俩如果收敛就必定收敛到相同的极限,将$b_n=2a_{n+1}-a_n$代入$b_n$的递推式中换掉$b_n$和$b_{n+1}$得到
  \[ (2a_{n+2}-a_{n+1})^{2} = a_{n+1}(2a_{n+1}-a_n) \]
  展开整理得
  \[ a_{n+2}^2-a_{n+2}a_{n+1} = \frac{1}{4}(a_{n+1}^2-a_{n+1}a_n) \]
  逐次把下标推下去,就有
  \[ a_{n+1}^2-a_{n+1}a_n = \frac{1}{4^{n+1}} \]
  把它视为关于$a_{n+1}$的二次方程,解之得
  \[ a_{n+1} = \frac{1}{2} \left( a_n+\sqrt{a_n^2+\frac{1}{4^n}} \right) \]
令$a_n=1/(2^n \tan{\theta_n})$,代入上式得$\tan{\theta_{n+1}}=\tan{\frac{\theta_n}{2}}$,所以可以取$\theta_{n+1}=\theta_n/2$,进而得$\theta_n=\pi / 2^n$,所以$n>1$时有
\[ a_n = \frac{1}{2^n \tan{\frac{\pi}{2^n}}} \]
比较$a_n$的递推式和定义式可得
\[ b_n = \sqrt{a_n^2+\frac{1}{4^n}} = \frac{1}{2^n \sin{\frac{\pi}{2^n}}} \]
由熟知的极限$\lim_{x \rightarrow 0} \frac{\sin{x}}{x}=1$即知$a_n$和$b_n$有共同的极限$1/\pi$.
\end{proof}

\begin{exercise}
  (isee)\footnote{原帖见 http://kuing.orzweb.net/viewthread.php?tid=4853}
  如果无穷数列$a_1,a_2,a_3,\ldots $对任意不同的两个正整数$i,j(i<j)$ 都有$|a_i-a_j| \geqslant \dfrac{1}{j}$,那么就称其为“难亲数列”,问题是难亲数列是否一定无界?
\end{exercise}

\begin{proof}[解答]
  初步想了下,好像是可以的,而且所有点都可以被限制在一个长度为2的闭区间上,当前$n$个点都确定后,$a_{n+1}$就不能落在前$n$个点以每一个点为中心半径为$\dfrac{1}{n+1}$的开邻域内,这些邻域的总长不超过$\dfrac{2n}{n+1}<2$,所以如果前$n$个点都被限制在一个长度为2的闭区间上的话,则这闭区间仍然有空隙可以容得下$a_{n+1}$,因此结论应该是可以的。

  再考虑构造出一个方法,让所有点都落在区间$[0,2]$上,首先让$a_1=0$,$a_2=2$,以后的点的确定,采用反复分割区间取中点的方法:

第一次分割前,区间数目为1,每个区间长度为2,按区间中点分割$a_3=1$,分割后区间数目为2,每个区间长度为1.

第二次分割前,区间数目为2,每个区间长度为1,取各个区间中点为分割$a_4=1/2$,$a_5=3/2$,分割后区间数目为$2^2$,每个区间长度为$\dfrac{1}{2}$.

第三次分割前,区间数目为$2^2$,每个区间长度为$\dfrac{1}{2}$,各个区间中点$a_6=1/4$,$a_7=3/4$,$a_8=5/4$,$a_9=7/4$,分割后区间数目为$2^3$,每个区间长度为$\dfrac{1}{2^2}$.

......

第$n+1$次分割前,区间数目为$2^n$,每个区间长度为$\dfrac{1}{2^{n-1}}$,各个区间中点$a_{2^n+2+i}=\dfrac{2i+1}{2^n}(i=0,1,\cdots,2^n-1)$,分割后区间数目为$2^{n+1}$,每个区间长度为$\dfrac{1}{2^n}$.

按此方法可以确定$a_n$,把$n$改写为$n=2^m+2+i(i=0,1,\cdots,2^m-1)$,于是$m=[\log_2(n-2)]$,$i=n-2^{[\log_2(n-2)]}$,这里中括号是向下取整,在这种表示下,就有
\[ a_{2^m+2+i}=\frac{2i+1}{2^m} \]
或者写成
\[ a_n=\frac{1+2(n-2-2^{[\log_2(n-2)]})}{2^{[\log_2(n-2)]}} \]
后者其实还不如前一种好看。

在这种分法下,$a_{2^m+2+i}(i=0,1,\cdots,2^m-1)$是第$m+1$次分割产生的分点,而在这一次分割后,每个区间的长度为$\dfrac{1}{2^m}$,所以到这一次分割后为止,所得到的$a_i$中的最大下标是$2^{m+1}+1$,即$i=2^m-1$的那个分点,此时任意两个$a_i$之差都$\geqslant \dfrac{1}{2^m}>\dfrac{1}{2^m+2+i}$,因此,由这分割所确定的数列符合要求。
\end{proof}


%%% Local Variables:
%%% mode: latex
%%% TeX-master: "../../book"
%%% End:
