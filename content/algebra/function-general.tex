
\section{函数概要}
\label{sec:function-general}


\subsection{函数概念}
\label{sec:function-concept}

函数是两个数集之间的一个映射,根据对应法则,数集$A$中每一个数在数集$B$中都有唯一一个数与之对应。函数通常写为$y=f(x)$,但这并不是说,所有函数都能表示成自变量的式子,比如黎曼函数就没有解析式,而隐函数$f(x,y)=0$甚至不能将$y$解成关于$x$的式子。

在讨论一些函数时,为了方便,将它的对应法则分解成嵌套的多个法则,于是得到复合函数的概念,但它并不是一类新的函数,只是认识函数对应法则的一个视角而已。

对于某些函数,由于它的对应法则的逆法则也正好满足函数定义(只要原法则下不存在多对一,则逆法则就不存在一对多,从而符合函数定义),因此自变量也就可以看成因变量的函数,这就是反函数,反函数与其原来函数是同一对应法则的两种表示方法,图象也是完全重合的,只有在互换$x$和$y$后,两者图象关于一三象限角平分线对称。

\subsection{函数的性质}
\label{sec:property-of-function}

比较通用的性质是单调性,对称性(含奇偶性),周期性,凸凹性等。

\subsubsection{单调性}
\label{sec:monotonicity-of-function}

\begin{definition}
  设函数$f(x)$在某个非空实数集$E$上有定义,如果对于任意$x_1,x_2 \in E$且$x_1<x_2$都有$f(x_1) \leqslant f(x_2)$,则称函数$f(x)$在$E$上是\emph{单调增加}的,如果不等式中的等号永远不成立,即总有$f(x_1)<f(x_2)$,则称函数$f(x)$在$E$上是\emph{严格增加}的,类似的可以得出在$E$上\emph{单调减少}和\emph{严格减少}的定义。
\end{definition}

单调性反应了两个变量的变化趋势,如果变化趋势一致,则为增函数,变化趋势相反则为减函数。但函数在某一区间上并不必然有某种单调性,有些函数无论你把区间划分得多么小,都没有单调性,比如狄利克雷函数和黎曼函数。

\begin{example}
这里讨论下函数$f(x)=x+\frac{a}{x}$的单调性,这里$a$是任何固定的正实数。

因为它是奇函数,奇函数在关于原点对称的区间上单调性情况相同,所以只要讨论$x>0$的情况即可,此时由于
$$
f(x)=\left( \sqrt{x}-\frac{\sqrt{a}}{\sqrt{x}} \right)^2+2
$$
括号中部分是关于$x$的增函数,但是外面有平方,还得考虑它的符号,在$x=\sqrt{a}$左侧为负右侧为正,所以$f(x)$在$(0,\sqrt{a}]$上单调减少,在$[\sqrt{a},+\infty)$上单调增加,在$x=\sqrt{a}$处有极小值$f(\sqrt{a})=2\sqrt{a}$,在$x$趋近于0和正无穷大时,函数值亦趋向于正无穷大,而且在这两个情形下,它分别与反比例函数$y=\frac{1}{x}$和正比例函数$y=x$无限逼近,因此它的图象如图所示。
\end{example}

\begin{example}
  我们已经知道指数为有理数的指数运算的定义,从而可以引申定义域为有理数集$\mathbb{Q}$上的指数函数,显然,如果$a>1$,则函数指数$a^x$在有理数集$\mathbb{Q}$上严格增加,相反,如果$0<a<1$,则是严格减少。
\end{example}

\subsubsection{对称性}
\label{sec:symmetric-of-function}

\begin{definition}
  如果函数$f(x)$的定义域$E$是对称数集(即若包含$x$,则也包含$-x$),且对于任意$x \in E$,有$f(-x)=-f(x)$,则称函数$f(x)$是定义在$E$上的\emph{奇函数},把这等式换成$f(-x)=f(x)$,则得到\emph{偶函数}的定义。
\end{definition}

奇偶性是对称性的特殊情况,更一般的情况是,若函数$f(x)$的图象关于直线$x=a$对称,则$f(a+x)=f(a-x)$,若它的图象关于点$(a,b)$中心对称,则$f(a+x)+f(a-x)=2b$。

\begin{theorem}
  定义在对称数集$E$上的任何一个函数$f(x)$,都能被表为该数集上的一个奇函数与一个偶函数之和。
\end{theorem}

\begin{proof}[证明]
  我们先假设$f(x)$能写成$E$上的奇函数$g(x)$与偶函数$h(x)$之和,即
  \begin{equation}
    \label{eq:56hjskaaa83}
   f(x)=g(x)+h(x) 
  \end{equation}
  于是有
  \[ f(-x)=g(-x)+h(-x) \]
  由于$g(-x)=-g(x)$,$h(-x)=h(x)$,所以
  \begin{equation}
    \label{eq:38dheo289}
   f(-x) = -g(x)+h(x) 
  \end{equation}
  把 \autoref{eq:56hjskaaa83}和 \autoref{eq:38dheo289}视为关于$g$和$h$的二元一次方程组,可以求得
  \[ g(x) = \frac{f(x)-f(-x)}{2}, \  h(x) = \frac{f(x)+f(-x)}{2} \]
  这样就实际求出了一个这样的表达,当然,这个表达式并非唯一。
\end{proof}

\subsubsection{周期性}
\label{sec:periodicity-of-function}

周期性反映了函数值重复取值的规律,三角函数的周期性人尽皆知。此处需要说明的是周期函数并不一定存在最小正周期,除了最为特殊的常量函数以外,狄利克雷函数(在任何有理数处函数值为1,而任何无理数处函数值为零)也可以说明这一点,任何有理数都是它的周期,而最小的正有理数是不存在的。

\subsubsection{凸性}
\label{sec:convex-property-of-function}

凸凹性反映了函数图像的拱形特征,这个性质是一大批不等式的本源,比如说,由对数函数的上凸性即得均值不等式,再由琴生(Jensen)不等式可推得多元均值不等式,更为宽泛的加权均值不等式仍然从对数函数的上凸性获得。本章有专门讨论这一性质的小节。

%%% Local Variables:
%%% mode: latex
%%% TeX-master: "../../elementary-math-note"
%%% End:
