
\section{代数学基础}
\label{sec:algebra-base}

\subsection{集合}
\label{sec:set}

集合是数学中不加定义的一个基本概念,一些确定的对象构成的整体即称为集合,构成集合的个体称为集合的元素,不包含任何元素的集合称为空集,记作$\emptyset$。

例如,平面上的所有直线构成一个集合,三维空间中的所有点构成一个集合。

所有的整数构成整数集$\mathbb{Z}$,全体有理数构成有理数集$\mathbb{Q}$,全体实数构成实数集$\mathbb{R}$,全体复数构成复数集$\mathbb{C}$.

如果集合中的全体元素都满足某个条件,并且满足此条件的元素都属于这集合,那么通常就用这条件来刻画这个集合,例如$A=\{ x | x \in \mathbb{Z}, x>0 \}$就表示正整数集合.

如果元素$x$是集合$A$中的元素,称为$x$属于集合$A$,记作$x \in A$,否则称$x$不属于集合$A$,记作$x \not in A$.

集合中的元素是不区分顺序的,如果两个集合$A$与$B$中的元素个数相同且对应相等,那么这两个集合称为相等,用符合语言就是:
\[ A = B \Leftrightarrow (x \in A \Rightarrow x \in B) \wedge (x in B \Rightarrow x in A) \]
上式中上尖括号$\wedge$表示逻辑上的与。
 规定所有的空集都是相等的。 

 相等是集合间的一种关系,除此之外,还有包含与被包含关系:
 \begin{definition}
   如果一个集合$B$中的任何元素都是集合$A$中的元素,则称$B$是$A$的\emph{子集},记作$B \subset B$,或者称$A$包含$B$并记作$A \supset B$,这两个称呼与符号是等价的.规定空集是任何集合(包括空集本身)的子集.
 \end{definition}

 显然, 相等是包含关系的特例,并且$A=B \Leftrightarrow (A \subset B) \wedge (B \subset A)$.

 如果$A$是$B$的子集且两者并不相等,则称$A$是$B$的\emph{真子集},显然,$A$是$B$的真子集的充分必要条件是,$A$是$B$的子集且$B$中至少包含一个不属于$A$的元素.
 
集合与集合之间可以进行一些运算,例如交、并、差等。
\begin{definition}
  设$A$与$B$是两个集合,由所有同时属于两者的元素所组成的集合称为它们的\emph{交集},记作$A \cap B$,即$A \cap B = \{ x | (x \in A) \wedge (x \in B) \}$.
\end{definition}

\begin{definition}
  设$A$与$B$是两个集合,由所有属于$A$或者属于$B$的元素所组成的集合称为它们的\emph{并集},记作$A \cup B$,即$A \cup B = \{x | (x \in A) \lor (x in B) \}$.
\end{definition}

交集与并集的混合运算满足德摩根律.
\begin{theorem}
  设$A$、$B$、$C$是三个集合,则
  \begin{align*}
    A \cap (B \cup C) & = (A \cap B) \cup (A \cap C) \\
    A \cup (B \cap C) & = (A \cup B) \cap (A \cup C) 
  \end{align*}
\end{theorem}

\begin{definition}
  设$A$与$B$是两个集合,由所有属于$A$但是不属于$B$的元素所组成集合称为$A$与$B$的\emph{差集},记作$A-B$,即$A-B = \{x | (x in A) \land (x \notin B) \}$.
\end{definition}

显然: $A-B = A - (A \cap B)$.

下面介绍集合间的一种重要的运算:笛卡尔积.
\begin{definition}
  设$A$、$B$是两个集合,则序偶$\langle a,b \rangle$的集合$\{ \pair{a}{b} | a \in A, b \in B \}$称为$A$与$B$的\emph{笛卡尔积},记作$A \times B$.
\end{definition}

注意,笛卡尔的元素是序偶,该序偶的第一个元素来自于集合$A$,而第二个元素来自于集合$B$.

显然,$A \times B$与$B \times A$是两个不同的集合。

\subsection{映射}
\label{sec:map}

\begin{definition}
  设$A$、$B$是两个非空集合,集合$C$是$A$与$B$的笛卡尔积$A \times B$的一个子集,如果集合$C$满足:
  \begin{enumerate}
  \item $\forall a \in A, \exists  b \in B \Rightarrow \pair{a}{b} \in C$.
  \item $ \pair{a}{b_1} \in C, \pair{a}{b_2} \in C \Rightarrow b_1=b_2$.
  \end{enumerate}
  则称集合$C$是$A$到$B$的一个\emph{映射},记作$f:A \mapsto B$.
\end{definition}

用通俗的语言说,从$A$到$B$的映射是一个对应关系,它使得$A$中任何元素(称为原象)都能对应到$B$中的某一个元素(称为象),并且不存在一对多的现象,但是多对一是允许的,并且$B$中可以有一些没有原象的元素。

定义中是映射的严格的符号化定义,可以看出,映射实际上是笛卡尔积的一个满足特定条件的一个子集.

函数就是从数集到数集的映射。

映射是可以进行运算的。
\begin{definition}
设$A$、$B$、$C$是三个非空集合,而$f$是从$A$到$B$的一个映射$f_{A \times B} = \{ \}$
\end{definition}

\subsection{运算与关系}
\label{sec:relation}




%%% Local Variables:
%%% mode: latex
%%% TeX-master: "../../elementary-math-note"
%%% End:
